%\pagenumbering{roman} \setcounter{page}{1}
%\chapter*{Abstract\markboth{Abstract}{Abstract}}

%\selectlanguage{spanish}
\begin{abstract}

A una red de sensores inalámbricos (WSN, por sus siglas en inglés) se le puede considerar un ambiente aislado si no existe un mecanismo de comunicación hacia el mundo exterior, por ejemplo Internet, por esta razón es necesario proveer a la WSN con un dispositivo ad-hoc que permita vincularla con otros sistemas heterogéneos de comunicación.

La utilidad de una red de sensores inalámbricos puede verse limitada sin la existencia de un mecanismo de comunicación hacia el exterior, por esta razón es necesario proveer a la red de un dispositivo ad-hoc que permita vincularla con otros sistemas heterogéneos de comunicación. 
 
Las redes de sensores inalámbricos WSN (Wireless Sensor Networks por su siglas en inglés) habilitan la interacción con el mundo físico mediante pequeños nodos que automáticamente se organizan para realizar operaciones aritméticas, lógicas, de sensado y de comunicación inalámbrica.  



Avances en la tecnología ha permitido el desarrollo de sensores cada vez más pequeños, de bajo consumo de energía y con comunicación inalámbrica que han encontrado aplicaciones en áreas como seguridad, monitoreo ambiente y rastro de objetos y personas. 

Puesto que la utilidad de una WSN puede verse limitada si no existe un mecanismo de comunicación hacia el mundo exterior, por ejemplo Internet, es necesario proveer a la WSN con un dispositivo ad hoc que permita vincularla con otros sistemas heterogéneos de comunicación, a este elemento se le conoce como data sink. 

Mediante el proyecto FOMIX-VER-2011-04-152366 el COVECyT y CONACyT encargaron al Laboratorio Nacional de Informática Avanzada (LANIA)
A.C. el desarrollo de una WSN para la detección temprana y el monitoreo en tiempo real de deslaves y desgajamientos de tierra por precipitación pluvial. Para esta aplicación se plantea el desarrollo de un dispositivo adaptador de protocolo que permita implementar un data sink utilizando una computadora industrial.

\end{abstract}